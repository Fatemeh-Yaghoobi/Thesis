% This file contains examples of various features from \AmS-\LaTeX{}.

Let $\mathbf{A}=(a_{ij})$ be the adjacency matrix of graph $G$. The
corresponding Kirchhoff matrix $\mathbf{K}=(k_{ij})$ is obtained from
$\mathbf{A}$ by replacing in $-\mathbf{A}$ each diagonal entry by the
degree of its corresponding vertex; i.e., the $i$th diagonal entry is
identified with the degree of the $i$th vertex. It is well known that
\begin{equation}
\det\mathbf{K}(i|i)=\text{ the number of spanning trees of $G$},
\quad i=1,\dots,n
\end{equation}
where $\mathbf{K}(i|i)$ is the $i$th principal submatrix of
$\mathbf{K}$.


Let $C_{i(j)}$ be the set of graphs obtained from $G$ by attaching edge
$(v_iv_j)$ to each spanning tree of $G$. Denote by $C_i=\bigcup_j
C_{i(j)}$. It is obvious that the collection of Hamiltonian cycles is a
subset of $C_i$. Note that the cardinality of $C_i$ is $k_{ii}\det
\mathbf{K}(i|i)$. Let $\wh X=\{\hat x_1,\dots,\hat x_n\}$.
Define multiplication for the elements of $\wh X$ by
\begin{equation}\label{multdef}
\hat x_i\hat x_j=\hat x_j\hat x_i,\quad \hat x^2_i=0,\quad
i,j=1,\dots,n.
\end{equation}
Let $\hat k_{ij}=k_{ij}\hat x_j$ and $\hat k_{ij}=-\sum_{j\not=i} \hat
k_{ij}$. Then the number of Hamiltonian cycles $H_c$ is given by the
relation 
\begin{equation}\label{H-cycles}
\biggl(\prod^n_{\,j=1}\hat x_j\biggr)H_c=\frac{1}{2}\hat k_{ij}\det
\wh{\mathbf{K}}(i|i),\qquad i=1,\dots,n.
\end{equation}
The task here is to express \eqref{H-cycles}
in a form free of any $\hat x_i$,
$i=1,\dots,n$. The result also leads to the resolution of enumeration of
Hamiltonian paths in a graph.
%
The resulting formula also involves the determinant and permanent, but
it can easily be applied to $K_n$ and $K_{n_1n_2}$. In addition, we
eliminate the permanent from $H_c$ and show that $H_c$ can be
represented by a determinantal function of multivariables, each variable
with domain $\{0,1\}$. Furthermore, we show that $H_c$ can be written by
number of spanning trees of subgraphs. Finally, we apply the formulas to
a complete multigraph $K_{n_1\dots n_p}$.

The conditions $a_{ij}=a_{ji}$, $i,j=1,\dots,n$, are not required in
this paper. All formulas can be extended to a digraph simply by
multiplying $H_c$ by 2.



\begin{notation} For $p,q\in P$ and $n\in\omega$ we write
$(q,n)\le(p,n)$ if $q\le p$ and $A_{q,n}=A_{p,n}$.
\end{notation}

Let $\mathbf{B}=(b_{ij})$ be an $n\times n$ matrix. Let $\mathbf{n}=\{1,
\dots,n\}$. Using the properties of \eqref{multdef}, it is readily seen
that

\begin{lem}\label{lem-per}
\begin{equation}
\prod_{i\in\mathbf{n}}
\biggl(\sum_{\,j\in\mathbf{n}}b_{ij}\hat x_i\biggr)
=\biggl(\prod_{\,i\in\mathbf{n}}\hat x_i\biggr)\per \mathbf{B}
\end{equation}
where $\per \mathbf{B}$ is the permanent of $\mathbf{B}$.
\end{lem}

Let $\wh Y=\{\hat y_1,\dots,\hat y_n\}$. Define multiplication
for the elements of $\wh Y$ by
\begin{equation}
\hat y_i\hat y_j+\hat y_j\hat y_i=0,\quad i,j=1,\dots,n.
\end{equation}

\begin{thm}\label{thm-main}
\begin{equation}\label{detB}
\det\mathbf{B}=
\sum^n_{l =0}\sum_{I_l \subseteq n}
\prod_{i\in I_l}(b_{ii}-\lambda_i)
\det\mathbf{B}^{(\lambda)}(I_l |I_l ),
\end{equation}
where $I_l =\{i_1,\dots,i_l \}$ and $\mathbf{B}^{(\lambda)}(I_l |I_l )$
is the principal submatrix obtained from $\mathbf{B}^{(\lambda)}$
by deleting its $i_1,\dots,i_l $ rows and columns.
\end{thm}

\begin{rem}
Let $\mathbf{M}$ be an $n\times n$ matrix. The convention
$\mathbf{M}(\mathbf{n}|\mathbf{n})=1$ has been used in \eqref{detB} and
hereafter.
\end{rem}


\begin{cor}\label{BI}
Write $\det(\mathbf{B}-x\mathbf{I})=\sum^n_{l =0}(-1)
^l b_l x^l $. Then
\begin{equation}\label{bl-sum}
b_l =\sum_{I_l \subseteq\mathbf{n}}\det\mathbf{B}(I_l |I_l ).
\end{equation}
\end{cor}
Let
\begin{equation}
\mathbf{K}(t,t_1,\dots,t_n)
=\begin{pmatrix} D_1t&-a_{12}t_2&\dots&-a_{1n}t_n\\
-a_{21}t_1&D_2t&\dots&-a_{2n}t_n\\
\hdotsfor[2]{4}\\
-a_{n1}t_1&-a_{n2}t_2&\dots&D_nt\end{pmatrix},
\end{equation}
where
\begin{equation}
D_i=\sum_{j\in\mathbf{n}}a_{ij}t_j,\quad i=1,\dots,n.
\end{equation}

Set
\begin{equation*}
D(t_1,\dots,t_n)=\frac{\delta}{\delta t}\eval{\det\mathbf{K}(t,t_1,\dots,t_n)
}_{t=1}.
\end{equation*}
Then
\begin{equation}\label{sum-Di}
D(t_1,\dots,t_n)
=\sum_{i\in\mathbf{n}}D_i\det\mathbf{K}(t=1,t_1,\dots,t_n; i|i),
\end{equation}
where $\mathbf{K}(t=1,t_1,\dots,t_n; i|i)$ is the $i$th principal
submatrix of $\mathbf{K}(t=1,t_1,\dots,t_n)$.

Theorem ~\ref{thm-main} leads to
\begin{equation}\label{detK1}
\det\mathbf{K}(t_1,t_1,\dots,t_n)
=\sum_{I\in\mathbf{n}}(-1)^{\envert{I}}t^{n-\envert{I}}
\prod_{i\in I}t_i\prod_{j\in I}(D_j+\lambda_jt_j)\det\mathbf{A}
^{(\lambda t)}(\overline{I}|\overline I).
\end{equation}
Note that
\begin{equation}\label{detK2}
\det\mathbf{K}(t=1,t_1,\dots,t_n)=\sum_{I\in\mathbf{n}}(-1)^{\envert{I}}
\prod_{i\in I}t_i\prod_{j\in I}(D_j+\lambda_jt_j)\det\mathbf{A}
^{(\lambda)}(\overline{I}|\overline{I})=0.
\end{equation}

Let $t_i=\hat x_i,i=1,\dots,n$. Lemma ~\ref{lem-per} yields
\begin{multline}
\biggl(\sum_{\,i\in\mathbf{n}}a_{l _i}x_i\biggr)
\det\mathbf{K}(t=1,x_1,\dots,x_n;l |l )\\
=\biggl(\prod_{\,i\in\mathbf{n}}\hat x_i\biggr)
\sum_{I\subseteq\mathbf{n}-\{l \}}
(-1)^{\envert{I}}\per\mathbf{A}^{(\lambda)}(I|I)
\det\mathbf{A}^{(\lambda)}
(\overline I\cup\{l \}|\overline I\cup\{l \}).
\label{sum-ali}
\end{multline}

We have
\begin{prop}\label{prop:eg}
\begin{equation}
H_c=\frac1{2n}\sum^n_{l =0}(-1)^{l}
D_{l},
\end{equation}
where
\begin{equation}\label{delta-l}
D_{l}=\eval[2]{\sum_{I_{l}\subseteq \mathbf{n}}
D(t_1,\dots,t_n)}_{t_i=\left\{\begin{smallmatrix}
0,& \text{if }i\in I_{l}\quad\\% \quad added for centering
1,& \text{otherwise}\end{smallmatrix}\right.\;,\;\; i=1,\dots,n}.
\end{equation}
\end{prop}



We consider here the applications of Theorems to a complete
multipartite graph $K_{n_1\dots n_p}$. It can be shown that the
number of spanning trees of $K_{n_1\dots n_p}$
may be written
\begin{equation}\label{e:st}
T=n^{p-2}\prod^p_{i=1}
(n-n_i)^{n_i-1}
\end{equation}
where
\begin{equation}
n=n_1+\dots+n_p.
\end{equation}

It follows from Theorems that
\begin{equation}\label{e:barwq}
\begin{split}
H_c&=\frac1{2n}
\sum^n_{{l}=0}(-1)^{l}(n-{l})^{p-2}
\sum_{l _1+\dots+l _p=l}\prod^p_{i=1}
\binom{n_i}{l _i}\\
&\quad\cdot[(n-l )-(n_i-l _i)]^{n_i-l _i}\cdot
\biggl[(n-l )^2-\sum^p_{j=1}(n_i-l _i)^2\biggr].\end{split}
\end{equation}
and
\begin{equation}\label{joe}
\begin{split}
H_c&=\frac12\sum^{n-1}_{l =0}
(-1)^{l}(n-l )^{p-2}
\sum_{l _1+\dots+l _p=l}
\prod^p_{i=1}\binom{n_i}{l _i}\\
&\quad\cdot[(n-l )-(n_i-l _i)]^{n_i-l _i}
\left(1-\frac{l _p}{n_p}\right)
[(n-l )-(n_p-l _p)].
\end{split}
\end{equation}

The enumeration of $H_c$ in a $K_{n_1\dotsm n_p}$ graph can also be
carried out by Theorem 
together with the algebraic method of \eqref{multdef}.
Some elegant representations may be obtained. For example, $H_c$ in
a $K_{n_1n_2n_3}$ graph may be written
\begin{equation}\label{j:mark}
\begin{split}
H_c=&
\frac{n_1!\,n_2!\,n_3!}
{n_1+n_2+n_3}\sum_i\left[\binom{n_1}{i}
\binom{n_2}{n_3-n_1+i}\binom{n_3}{n_3-n_2+i}\right.\\
&+\left.\binom{n_1-1}{i}
\binom{n_2-1}{n_3-n_1+i}
\binom{n_3-1}{n_3-n_2+i}\right].\end{split}
\end{equation}




\begin{defn}

A function $H\colon \Re^n \to \Re^n$ is said to be
\emph{B-differentiable} at the point $z$ if (i)~$H$ is Lipschitz
continuous in a neighborhood of $z$, and (ii)~ there exists a positive
homogeneous function $BH(z)\colon \Re^n \to \Re^n$, called the
\emph{B-derivative} of $H$ at $z$, such that
\[ \lim_{v \to 0} \frac{H(z+v) - H(z) - BH(z)v}{\enVert{v}} = 0. \]
The function $H$ is \textit{B-differentiable in set $S$} if it is
B-differentiable at every point in $S$. The B-derivative $BH(z)$ is said
to be \textit{strong} if
\[ \lim_{(v,v') \to (0,0)} \frac{H(z+v) - H(z+v') - BH(z)(v
 -v')}{\enVert{v - v'}} = 0. \]
\end{defn}


\begin{lem}\label{limbog} There exists a smooth function $\psi_0(z)$
defined for $\abs{z}>1-2a$ satisfying the following properties\textup{:}
\begin{enumerate}
\renewcommand{\labelenumi}{(\roman{enumi})}
\item $\psi_0(z)$ is bounded above and below by positive constants
$c_1\leq \psi_0(z)\leq c_2$.
\item If $\abs{z}>1$, then $\psi_0(z)=1$.
\item For all $z$ in the domain of $\psi_0$, $\Delta_0\ln \psi_0\geq 0$.
\item If $1-2a<\abs{z}<1-a$, then $\Delta_0\ln \psi_0\geq
c_3>0$.
\end{enumerate}
\end{lem}


Let $t\in\mathbf{R}$ be such that
$\abs{\wt{D}u}(\interval{\left[t,s\right[})>0$ for every $s>t$ and
assume that the limits in \eqref{joe} exist. By \eqref{j:mark} we get
\begin{equation*}\begin{split}
\frac{\hat v(s)-\hat
v(t)}{\abs{\wt{D}u}(\interval{\left[t,s\right[})}&=\frac {f(\hat
u(s))-f(\hat u(t))}{\abs{\wt{D}u}(\interval{\left[t,s\right[})}\\
&=\frac{f(\hat u(s))-f(\hat
u(t)+\dfrac{\wt{D}u}{\abs{\wt{D}u}}(t)\abs{\wt{D}u
}(\interval{\left[t,s\right[}))}%
{\abs{\wt{D}u}(\interval{\left[t,s\right[})}\\
&+\frac
{f(\hat u(t)+\dfrac{\wt{D}u}{\abs{\wt{D}u}}(t)\abs{\wt{D}
u}(\interval{\left[t,s\right[}))-f(\hat
u(t))}{\abs{\wt{D}u}(\interval{\left[t,s\right[})}
\end{split}\end{equation*}
for every $s>t$. Using the Lipschitz condition on $f$ we find
{\setlength{\multlinegap}{0pt}
\begin{multline*}
\left\lvert\frac{\hat v(s)-\hat
v(t)}{\abs{\wt{D}u}(\interval{\left[t,s\right[})} -\frac{f(\hat
u(t)+\dfrac{\wt{D}u}{\abs{\wt{D}u}}(t)
\abs{\wt{D}u}(\interval{\left[t,s\right[}))-f(\hat
u(t))}{\abs{\wt{D}u}(\interval{\left[t,s\right[})}\right\rvert\\
\le K\left\lvert
\frac{\hat u(s)-\hat u(t)}
  {\abs{\wt{D}u}(\interval{\left[t,s\right[})}
-\frac{\wt{D}u}{\abs{
\wt{D}u}}(t)\right\rvert.\end{multline*}
}% end of group with \multlinegap=0pt



\subsection{Bold versions of special symbols}

In the \pkg{amsmath} package \cn{boldsymbol} is used for getting
individual bold math symbols and bold Greek letters---everything in
math except for letters of the Latin alphabet,
where you'd use \cn{mathbf}.  For example,
looks like this:
\[A_\infty + \pi A_0 \sim \mathbf{A}_{\boldsymbol{\infty}}
\boldsymbol{+} \boldsymbol{\pi} \mathbf{A}_{\boldsymbol{0}}\]

\subsection{``Poor man's bold''}
If a bold version of a particular symbol doesn't exist in the
available fonts,
then \cn{boldsymbol} can't be used to make that symbol bold.
At the present time, this means that
\cn{boldsymbol} can't be used with symbols from
the \fn{msam} and \fn{msbm} fonts, among others.
In some cases, poor man's bold (\cn{pmb}) can be used instead
of \cn{boldsymbol}:
%  Can't show example from msam or msbm because this document is
%  supposed to be TeXable even if the user doesn't have
%  AMSFonts.  MJD 5-JUL-1990
\[\frac{\partial x}{\partial y}
\pmb{\bigg\vert}
\frac{\partial y}{\partial z}\]
So-called ``large operator'' symbols such as $\sum$ and $\prod$
require an additional command, \cn{mathop},
to produce proper spacing and limits when \cn{pmb} is used.
For further details see \textit{The \TeX book}.
\[\sum_{\substack{i<B\\\text{$i$ odd}}}
\prod_\kappa \kappa F(r_i)\qquad
\mathop{\pmb{\sum}}_{\substack{i<B\\\text{$i$ odd}}}
\mathop{\pmb{\prod}}_\kappa \kappa(r_i)
\]


\subsection{Multiple integral signs}

\cn{iint}, \cn{iiint}, and \cn{iiiint} give multiple integral signs
with the spacing between them nicely adjusted,  in both text and
display style.  \cn{idotsint} gives two integral signs with dots
between them.
\begin{gather}
\iint\limits_A f(x,y)\,dx\,dy\qquad\iiint\limits_A
f(x,y,z)\,dx\,dy\,dz\\
\iiiint\limits_A
f(w,x,y,z)\,dw\,dx\,dy\,dz\qquad\idotsint\limits_A f(x_1,\dots,x_k)
\end{gather}

\subsection{Over and under arrows}

Some extra over and under arrow operations are provided in
the \pkg{amsmath} package.  (Basic \LaTeX\ provides
\cn{overrightarrow} and \cn{overleftarrow}).
\begin{align*}
\overrightarrow{\psi_\delta(t) E_t h}&
=\underrightarrow{\psi_\delta(t) E_t h}\\
\overleftarrow{\psi_\delta(t) E_t h}&
=\underleftarrow{\psi_\delta(t) E_t h}\\
\overleftrightarrow{\psi_\delta(t) E_t h}&
=\underleftrightarrow{\psi_\delta(t) E_t h}
\end{align*}
These all scale properly in subscript sizes:
\[\int_{\overrightarrow{AB}} ax\,dx\]


\subsection{Operator names}
The more common math functions such as $\log$, $\sin$, and $\lim$
have predefined control sequences: \verb=\log=, \verb=\sin=,
\verb=\lim=.
The \pkg{amsmath} package provides \cn{DeclareMathOperator} and
\cn{DeclareMathOperator*}
for producing new function names that will have the
same typographical treatment.
Examples:
\[\norm{f}_\infty=
\esssup_{x\in R^n}\abs{f(x)}\]

\[\meas_1\{u\in R_+^1\colon f^*(u)>\alpha\}
=\meas_n\{x\in R^n\colon \abs{f(x)}\geq\alpha\}
\quad \forall\alpha>0.\]

\cn{esssup} and \cn{meas} would be defined in the document preamble as


The following special operator names are predefined in the \pkg{amsmath}
package: \cn{varlimsup}, \cn{varliminf}, \cn{varinjlim}, and
\cn{varprojlim}. Here's what they look like in use:
\begin{align}
&\varlimsup_{n\rightarrow\infty}
       \mathcal{Q}(u_n,u_n-u^{\#})\le0\\
&\varliminf_{n\rightarrow\infty}
  \left\lvert a_{n+1}\right\rvert/\left\lvert a_n\right\rvert=0\\
&\varinjlim (m_i^\lambda\cdot)^*\le0\\
&\varprojlim_{p\in S(A)}A_p\le0
\end{align}


\subsection{Continued fractions}
The continued fraction
\begin{equation}
\cfrac{1}{\sqrt{2}+
 \cfrac{1}{\sqrt{2}+
  \cfrac{1}{\sqrt{2}+
   \cfrac{1}{\sqrt{2}+
    \cfrac{1}{\sqrt{2}+\dotsb
}}}}}
\end{equation}
can be obtained by typing


\subsection{The `cases' environment}
`Cases' constructions like the following can be produced using
the \env{cases} environment.
\begin{equation}
P_{r-j}=
  \begin{cases}
    0&  \text{if $r-j$ is odd},\\
    r!\,(-1)^{(r-j)/2}&  \text{if $r-j$ is even}.
  \end{cases}
\end{equation}

Notice the use of \cn{text} and the embedded math.

\subsection{Matrix}

Here are samples of the matrix environments,
\cn{matrix}, \cn{pmatrix}, \cn{bmatrix}, \cn{Bmatrix}, \cn{vmatrix}
and \cn{Vmatrix}:
\begin{equation}
\begin{matrix}
\vartheta& \varrho\\\varphi& \varpi
\end{matrix}\quad
\begin{pmatrix}
\vartheta& \varrho\\\varphi& \varpi
\end{pmatrix}\quad
\begin{bmatrix}
\vartheta& \varrho\\\varphi& \varpi
\end{bmatrix}\quad
\begin{Bmatrix}
\vartheta& \varrho\\\varphi& \varpi
\end{Bmatrix}\quad
\begin{vmatrix}
\vartheta& \varrho\\\varphi& \varpi
\end{vmatrix}\quad
\begin{Vmatrix}
\vartheta& \varrho\\\varphi& \varpi
\end{Vmatrix}
\end{equation}
%

To produce a small matrix suitable for use in text, use the
\env{smallmatrix} environment.

To show
the effect of the matrix on the surrounding lines of
a paragraph, we put it here: \begin{math}
  \bigl( \begin{smallmatrix}
      a&b\\ c&d
    \end{smallmatrix} \bigr)
\end{math}
and follow it with enough text to ensure that there will
be at least one full line below the matrix.

\cn{hdotsfor}\verb"{"\textit{number}\verb"}" produces a row of dots in a matrix
spanning the given number of columns:
\[W(\Phi)= \begin{Vmatrix}
\dfrac\varphi{(\varphi_1,\varepsilon_1)}&0&\dots&0\\
\dfrac{\varphi k_{n2}}{(\varphi_2,\varepsilon_1)}&
\dfrac\varphi{(\varphi_2,\varepsilon_2)}&\dots&0\\
\hdotsfor{5}\\
\dfrac{\varphi k_{n1}}{(\varphi_n,\varepsilon_1)}&
\dfrac{\varphi k_{n2}}{(\varphi_n,\varepsilon_2)}&\dots&
\dfrac{\varphi k_{n\,n-1}}{(\varphi_n,\varepsilon_{n-1})}&
\dfrac{\varphi}{(\varphi_n,\varepsilon_n)}
\end{Vmatrix}\]

The spacing of the dots can be varied through use of a square-bracket
option, for example, \verb"\hdotsfor[1.5]{3}".  The number in square brackets
will be used as a multiplier; the normal value is 1.

\subsection{The \cn{substack} command}

The \cn{substack} command can be used to produce a multiline
subscript or superscript:
for example

produces a two-line subscript underneath the sum:
\begin{equation}
\sum_{\substack{0\le i\le m\\ 0<j<n}} P(i,j)
\end{equation}
A slightly more generalized form is the \env{subarray} environment which
allows you to specify that each line should be left-aligned instead of
centered, as here:
\begin{equation}
\sum_{\begin{subarray}{l}
        0\le i\le m\\ 0<j<n
      \end{subarray}}
 P(i,j)
\end{equation}



\subsection{Split}
The \env{split} environment is not an independent environment
but should be used inside something else such as \env{equation}
or \env{align}.

If there is not enough room for it, the equation number for  a
\env{split} will be shifted to the previous line, when equation numbers are
on the left; the number shifts down to the next line when numbers are on
the right.
\begin{equation}
\begin{split}
f_{h,\varepsilon}(x,y)
&=\varepsilon\mathbf{E}_{x,y}\int_0^{t_\varepsilon}
L_{x,y_\varepsilon(\varepsilon u)}\varphi(x)\,du\\
&= h\int L_{x,z}\varphi(x)\rho_x(dz)\\
&\quad+h\biggl[\frac{1}{t_\varepsilon}\biggl(\mathbf{E}_{y}
  \int_0^{t_\varepsilon}L_{x,y^x(s)}\varphi(x)\,ds
  -t_\varepsilon\int L_{x,z}\varphi(x)\rho_x(dz)\biggr)\\
&\phantom{{=}+h\biggl[}+\frac{1}{t_\varepsilon}
  \biggl(\mathbf{E}_{y}\int_0^{t_\varepsilon}L_{x,y^x(s)}
    \varphi(x)\,ds -\mathbf{E}_{x,y}\int_0^{t_\varepsilon}
   L_{x,y_\varepsilon(\varepsilon s)}
   \varphi(x)\,ds\biggr)\biggr]\\
&=h\wh{L}_x\varphi(x)+h\theta_\varepsilon(x,y),
\end{split}
\end{equation}
Some text after to test the below-display spacing.


Use of \env{split} within \env{align}:
{\delimiterfactor750
\begin{align}
\begin{split}\abs{I_1}
  &=\left\lvert \int_\Omega gRu\,d\Omega\right\rvert\\
&\le C_3\left[\int_\Omega\left(\int_{a}^x
  g(\xi,t)\,d\xi\right)^2d\Omega\right]^{1/2}\\
&\quad\times \left[\int_\Omega\left\{u^2_x+\frac{1}{k}
  \left(\int_{a}^x cu_t\,d\xi\right)^2\right\}
  c\Omega\right]^{1/2}\\
&\le C_4\left\lvert \left\lvert f\left\lvert \wt{S}^{-1,0}_{a,-}
  W_2(\Omega,\Gamma_l)\right\rvert\right\rvert
  \left\lvert \abs{u}\overset{\circ}\to W_2^{\wt{A}}
  (\Omega;\Gamma_r,T)\right\rvert\right\rvert.
\end{split}\label{eq:A}\\
\begin{split}\abs{I_2}&=\left\lvert \int_{0}^T \psi(t)\left\{u(a,t)
  -\int_{\gamma(t)}^a\frac{d\theta}{k(\theta,t)}
  \int_{a}^\theta c(\xi)u_t(\xi,t)\,d\xi\right\}dt\right\rvert\\
&\le C_6\left\lvert \left\lvert f\int_\Omega
 \left\lvert \wt{S}^{-1,0}_{a,-}
  W_2(\Omega,\Gamma_l)\right\rvert\right\rvert
  \left\lvert \abs{u}\overset{\circ}\to W_2^{\wt{A}}
  (\Omega;\Gamma_r,T)\right\rvert\right\rvert.
\end{split}
\end{align}}%
Some text after to test the below-display spacing.

\subsection{Multline}
Numbered version:
\begin{multline}\label{eq:E}
\int_a^b\biggl\{\int_a^b[f(x)^2g(y)^2+f(y)^2g(x)^2]
 -2f(x)g(x)f(y)g(y)\,dx\biggr\}\,dy \\
 =\int_a^b\biggl\{g(y)^2\int_a^bf^2+f(y)^2
  \int_a^b g^2-2f(y)g(y)\int_a^b fg\biggr\}\,dy
\end{multline}
To test the use of \verb=\label= and
\verb=\ref=, we refer to the number of this
equation here: (\ref{eq:E}).

%%%%%%%%%%%%%%%%%%%%%%%%%%%

%%%%%%%%%%%%%%%%%%%%%%%%%%%%%%%%%%%%%%%%%%%%%%%%%%%%%

The \verb"*"-ed form of \env{gather} with the non-\verb"*"-ed form of
\env{align}.
\begin{gather*}
\begin{split} \varphi(x,z)
&=z-\gamma_{10}x-\gamma_{mn}x^mz^n\\
&=z-Mr^{-1}x-Mr^{-(m+n)}x^mz^n
\end{split}\\[6pt]
\begin{align} \zeta^0&=(\xi^0)^2,\\
\zeta^1 &=\xi^0\xi^1,\\
\zeta^2 &=(\xi^1)^2,
\end{align}
\end{gather*}
Some text after to test the below-display spacing.



%%%%%%%%%%%%%%%%%%%%%%%%%%%
\subsection{Blackboard and Calligraphic math}
$\widehat{\mathbb{Z}}$
